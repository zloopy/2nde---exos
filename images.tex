\documentclass[12pt,a4paper]{article}
\usepackage[left=2cm,right=2cm,top=2cm,bottom=3cm]{geometry}
\usepackage{amsmath,amsfonts,amsthm,amssymb,varioref,times}



%pour ecrire en français avec les accents
\usepackage[utf8]{inputenc}
\usepackage[T1]{fontenc}
\usepackage{lmodern} % load a font with all the characters
\usepackage{units}

%pour faire des cadres
\usepackage{framed}
\usepackage[dvipsnames]{xcolor}
\usepackage{tcolorbox}
\usepackage{tikz}
\usepackage{tkz-euclide}
\usetkzobj{all}

%chemistry frmulae
\usepackage{chemfig}
\usepackage{chemformula}

\begin{document}

\begin{tikzpicture}[scale=0.8]

% baseline
\draw (1,0) -- (18,0) ; 

%triangle 
\draw[very thick] (5,0) -- (5,10) -- (10,0) -- (5,0); 
\draw (5,0) node [below]{$A$} ; 
\draw (5,10) node [above]{$C$} ; 
\draw (10,0) node [below]{$B$} ; 
\draw (16,0) node [below]{$D$} ; 
\draw (5,4.36) node [right]{$I$} ; 
\draw (8.5,4) node [right]{$J$} ; 
\draw (5,0) rectangle (5.5,0.5) ; 

%rayon lumineux
\draw[dashed, ->] (1,4) -- (2,4) ; 
\draw[dashed, ->] (2,4) -- (5,4) -- (7.9,4) -- (16,0) ;  

%angles 
\draw (4.75,4) rectangle (5,4.25) ; 
\draw (14.9,0.5) arc (150:180:1cm);
\draw (5,9) arc (270:298:1cm) ; 



\end{tikzpicture}











\end{document}