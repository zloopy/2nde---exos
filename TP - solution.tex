\documentclass[12pt,a4paper]{article}
%\documentclass[12pt,a4paper]{article}
\usepackage[left=2cm,right=2cm,top=2cm,bottom=3cm]{geometry}
\usepackage{amsmath,amsfonts,amsthm,amssymb,varioref,times}
\usepackage{gensymb}
 \usepackage[explicit]{titlesec}
    % Raised Rule Command:
    % Arg 1 (Optional) - How high to raise the rule
    % Arg 2 - Thickness of the rule
    \newcommand{\raisedrulefill}[2][0ex]{\leaders\hbox{\rule[#1]{0.5pt}{#2}}\hfill}
    \titleformat{\section}{\Large\bfseries}{\thesection. }{0em}{#1\,\raisedrulefill[.5ex]{0.5pt}}

%to resume numbering in a list
\usepackage{enumitem}

%pour ecrire en français avec les accents
\usepackage[utf8]{inputenc}
\usepackage[T1]{fontenc}
\usepackage{lmodern} % load a font with all the characters
\usepackage{units}

%Image-related packages
\usepackage{wrapfig}
\usepackage{float, graphicx}
\graphicspath{ {./img/} }
\usepackage{subcaption}
\usepackage[export]{adjustbox}


%pour faire des cadres
\usepackage{framed}
\usepackage[dvipsnames]{xcolor}
\usepackage{tcolorbox}
\usepackage{tikz}

%chemistry frmulae
\usepackage{chemfig}
\usepackage{chemformula}
 
% pour ecrire sur +sieurs colonnes
\usepackage{multicol}
\setlength{\columnseprule}{0.25pt}
\setlength{\columnsep}{60pt}

% Fusion de lignes de tableaux.
\usepackage{multirow}

% Position verticale des lettres dans la ligne de tableau.
\usepackage{array}


% MATH -----------------------------------------------------------
\newcommand{\To}{\longrightarrow}
\newcommand{\gpl}{\; g\cdot L^{-1}}
\newcommand{\gpmol}{\; g\cdot mol^{-1}}
\newcommand{\mpl}{\; mol\cdot L^{-1}}
\newcommand{\mps}{\; m\cdot s^{-1}}
\newcommand{\mpss}{\; m\cdot s^{-2}}
\newcommand{\es}[1]{\cdot10^{#1}}
\newcommand{\eng}[1]{(\textcolor{purple}{= #1})}
\newcommand{\norm}[1]{\left\Vert#1\right\Vert}
\newcommand{\abs}[1]{\left\vert#1\right\vert}

\newenvironment{eg}
 {\begin{shaded} \textbf{Exemple:} } { \end{shaded}}

\newcounter{exo}
\newenvironment{exo}[1][]
 {\refstepcounter{exo} \begin{shaded}\noindent $\triangleright \quad$\textbf{Exercice~\theexo. #1} } { \end{shaded}}    

\newenvironment{defn}[1]
 {\begin{leftbar}\noindent \textbf{Définition :\textit{ \quad #1} } } { \end{leftbar}} 
\newenvironment{rmrq}
 {\begin{shaded} \textbf{Remarque: Pour aller plus loin ...}\\ \itshape } { \end{shaded}}

\definecolor{shadecolor}{gray}{0.9}



% parametres des entete et de pieds de pages
\usepackage{fancyhdr}
\pagestyle{fancy}
\fancyhf{}
\lhead{SciPhy : 2nde}
\rhead{Ch. 12 - Systèmes Optiques}
\chead{2019}
\rfoot{Page \thepage}
\lfoot{SZayyani}
\title{Chapitre 12 - Systèmes optiques}
\date{}
\author{}

\setlength{\parindent}{0mm}
\setlength{\parskip}{2mm}


%\begin{document}
%\maketitle
%\vspace{-1.9cm}

%\end{document}


% parametres des entete et de pieds de pages
\usepackage{fancyhdr}
\pagestyle{fancy}
\fancyhf{}
\lhead{SciPhy : 2nde}
\rhead{TP - Solutions}
%\chead{2019-27}
\rfoot{Page \thepage}
\lfoot{SZayyani}

\title{TP - Préparation d'une Solution}
\date{}
\author{}

\setlength{\parindent}{0mm}
\setlength{\parskip}{2mm}

%%%%%%%%%%% For wrapfigure 
\setlength{\intextsep}{6pt}%
\setlength{\columnsep}{3pt}%

\usepackage{setspace}
\doublespacing

\begin{document}
\maketitle
\vspace{-3cm}

\section{Préparation d'une Solution}
\subsection*{Problème : }
Dans l'une de ses recettes, l'apprenti Schtroumpf doit utiliser $100\; mL$ d'une solution aqueuse de sulfate de cuivre ($\ch{CuSO4}$, $5\ch{H2O}$) de concentration molaire $1,0\es{-1}\mpl$.

Dans son laboratoire, il ne possède que du sulfate de cuivre solide sous forme de poudre et d'une fiole jaugée de $100\; mL$. Il a besoin de votre ingéniosité pour préparer cette solution.

\subsection*{Questions : }
\begin{enumerate}
    \item Décrire le protocole expérimental que vous allez mettre en \oe vre afin d'aider le Schtroumpf à réaliser son projet. 
    
    (Donner les étapes, numérotées. N'hésitez pas d'inclure des schémas pour illustrer certaines étapes. )
    \item Faites vérifier votre solution le professeur, et puis avec son accord, réalisez-le. 
\end{enumerate}
\begin{tcolorbox}[title=Données : Masses molaires ($g\cdot mol^{-1} $)]
$M(Cu) = 63,5$ ;  $M(S) = 32,1 ;  M(O) = 16,0 ; M(H) = 1,0$, 
%\tcblower
\end{tcolorbox}

\begin{enumerate}[resume]
    \item Un Schtroumpf a, malencontreusement,  versé trop d’eau dans le récipient.  Peut-il enlever de la solution pour compenser l’erreur ? Pourquoi ?
\end{enumerate}

\section{Dilution de la solution}
Le Schtroumpf curieux doit également utiliser $V_f = 100\; mL$ d'une solution aqueuse $F$ de sulfate de cuivre de concentration molaire $[\ch{CuSO4}]_f = 1,0\es{-2}\mpl$. Or il ne dispose que d'une solution aqueuse qu'il a préparé précédemment de concentration $[\ch{CuSO4}]_m = 1,0\es{-1}\mpl$ et d'une fiole jaugée de $100\; mL$. 

\subsection*{Questions : }
\begin{enumerate}[resume]
    \item Déterminez la quantité de matière $n_f$ de sulfate de cuivre dans $v_f = 100\; mL$ de la solution $F$ de concentration $[\ch{CuSO4}]_F$.
    \item Déterminez le volume $v_m$ de la solution $M$ qui contient la même quantité de matière de sulfate de cuivre que $n_f$. 
    \item Proposez un protocole expérimental pour préparer la solution aqueuse $F$ de sulfate de cuivre à partir de la solution $M$. 
    Comme avant, le protocole s'écrit en plusieurs étapes numérotées, et avec des schémas quand nécessaire. 
    \item Faites vérifier votre protocole par le professeur, et puis avec son accord, réalisez-le. 
\end{enumerate}

\end{document}